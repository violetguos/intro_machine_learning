\documentclass[letterpaper, 12]{article}

%% Language and font encodings
\usepackage[english]{babel}
\usepackage[utf8x]{inputenc}
\usepackage[T1]{fontenc}

%% Sets page size and margins
\usepackage[letterpaper,top=2.5cm,bottom=2cm,left=2cm,right=2cm,marginparwidth=1.75cm]{geometry}

%% Useful packages
\usepackage{amsmath}
\usepackage{amssymb}
\usepackage{amsfonts}
\usepackage{graphicx}
\usepackage{physics}
\usepackage{bbold}
\usepackage[colorinlistoftodos]{todonotes}
\usepackage[colorlinks=true, allcolors=blue]{hyperref}
\usepackage{listings}
\usepackage{multicol}
\usepackage{float}
\usepackage{enumitem}

\usepackage{bm}
\date{\today}

\title{CSC411 Assignment 2}
\author{Yue Guo}
\begin{document}
\maketitle

%\centering
%  \includegraphics[width=0.5\textwidth]{1.3.2/1_3_2_k50.png}
%  \caption{k-NN Regression of $x \in [0, 11]$ with $k$ = 50.}
%\end{figure}


%%%%%%%%%%%Q3%%%%%%%%%%%%%%
\section{Kernels}
%%%%%%%%%% q 3.1%%%%%%%%%%%%%%
\subsection{Positive semidefinite and quadratic form}
Assume K is symmetric, we can decompose K into $U \Lambda U^T$
\begin{equation*}
\begin{split}
x^T K x &= x^T (U \Lambda U^T) x = (x^T U) \Lambda (U^T x)\\
&= \Sigma_{i =1}^{d} \lambda_{i} ([x^T U_{i}])^2 >= 0\\
\end{split}
\end{equation*}

%%%%%%%%%Q 3.2 %%%%%%%%%%%%%%
\subsection{Kernel properties}
%%%%%%%% Q 3.2 q2%%%%%%%%%%%%%
 \subsubsection{$\alpha$}
 K_{ij} = \alpha, the matrix K has dimension of x or y, and each element is \alpha. Since \alpha > 0, and all elements are eqaul, K is positive semidefinite
 
 \subsubseciton{f(x), f(y)}
 


\end{document}